\documentclass{article} 
\usepackage{geometry}
\geometry{
 a4paper,
 total={170mm,257mm},
 left=20mm,
 top=20mm,
 }

\title{ORIE 4741 Project Midterm Report}

\author{
  D'Azevedo, Gloria\\
  \texttt{gad87@cornell.edu}
  \and
  Yadav, Pihu\\
  \texttt{py82@cornell.edu}
}
\date{October 28, 2016}

\begin{document}
\maketitle

\section{Data Cleaning Remarks}
During the following report we will take on the convention that a "participant" is the female and the "partner" is the male.  These conventions are used when scraping data and analyzing the coefficients of the predictors in the model so that the data is not double counted.  All females have interacted with each male once and vice versa so this is not a problem.  There is only one incident of a person not completing her survey, and we address that issue below.

At the beginning of the analysis, we noted that some cities and jobs had commas in them, so had to remove all those otherwise the comma delimiter during the import would break up all those fields and not import the correct value per column or the correct number of columns.  We also removed apostrophe's  or rewrote the word so that it wouldn't need to be shortened with an apostrophe (i.e. changing "Int'l" to "International") since that seemed to cause import errors.\\

There were some blanks in the "pid" field that corresponded to the partner's id of the current participant for some of the id's in Wave 5. We deduced that the 7th female  didn't complete her responses to either the first survey or to the first follow up survey. Thus, we assume that she did not match with anyone and assign the "dec" and "dec\_0" fields which correspond to her decision to be 0. We chose not to delete all the corresponding rows entirely since they have information about male preferences for not matching with her.\\

For the "int\_corr" field which is the correlation between the participant and the partner's ratings of interesting Time 1, we substitute the blanks for 0's to indicate that there is no correlation as we are not sure how the field is exactly calculated. If we decide to use a similar field, we can insert a new "correlation" column at the end of the data where we know the calculation deterministically.\\

Some blank fields are inter-related so if we don't have values in one of them, then we cannot look up the value in another. For example, there are two fields "age\_o" and "age". "Age" is the self-reported age of the student and is asked when they signup. "age\_o" is the age of the partner during that round. The problem with age is that there are many possibilities that people did not report their age.  For example, men tend to like young women, so women may report a lower age than the true value.  On the other hand, women prefer older, mature men as a measure of stability so men may report a higher age than the true value.  For now we did not edit or add any of the "age" or "age\_o" values.\\

While investigating trends in the "field\_cd" column, which corresponds to a numerical code for the field of study that each person has, we noted that there were some NA values  so we reassigned those NA values to be 18, which is the "Other" field.  In addition, we also found that in some of the values in the "race" field were "NA" so we recoded those to be 6 which corresponds to the Other category.\\

For some of the fields, when a participant was asked to rate their preferences about their partner, there are different scales for this variable in different waves.  For example, during the initial survey before the events took place, the participants in waves 1-5 and 10-21 were asked to divide up 100 points among 6 different categories.  In contrast, the participants in waves 6-9 were asked to rate each category on a scale of 1-10 for the importance of each attribute in a partner (a rating of 1 indicates that the attribute is not at all important while a rating of 10 is extremely important).  We created new columns for these attributes using a normalized value so that they can be compared more accurately.

\section{Initial Findings and Ideas}
A suggestion from an outside source suggested that we create a field that groups the institutions that the students attended for their undergraduate degree together and see if that is a significant factor to whether or not two people will match.  We note that this field is pretty messy and has many blanks or NA's (around 40\% overall) so we do not think that we can reliably use this field as a predictor but potentially it can be used as a tie breaker when testing the model on new data.\\

We note that for some of the fields, they ask the participant to what extent other people perceive as important traits. However, it may not be that important to know what other people think--it's better to look at actual proof of what people like based on the people that they said they would match with them.  It may also be useful to look at what traits that they liked or preferred in their partner, regardless of whether or the other person actually liked them back.\\

%Importance of activities: 
Note that there are around 17 different activities that participants are asked to 
write their amount of interest in them.  These values are integer values from 1 to 10.  If they have similar amounts of interest in the activity, then we assume that they will both rank the activity similarly and we exclude the possibility that the participants mis-rank the interest level (range of ranks are from 1 to 10, 1 being not at all important, 10 being extremely important).\\

We consider a norm of a vector of differences in interest that a participant and her partner have in an activity.  Using the norm squared versus another error measure such as absolute value will penalize larger differences more, although the maximum amount is 81 in this case for each activity. Using the absolute value (or the 1-norm) will penalize at most 9 for each activity. Then later on, maybe we can use this value as another field in the data instead of using the activities separately. \\
 
 \section{Testing Initial Models}
 

\end{document}